\documentclass[dvipdfmx,a4paper,12pt]{jsarticle}

\usepackage{amsmath,amssymb,amsthm}
\usepackage{verbatim}
\usepackage{graphicx}
\usepackage{url}
\usepackage[dvipdfmx,hidelinks]{hyperref}
\usepackage{pxjahyper}
\usepackage{color}
\definecolor{TiffanyBlue}{RGB}{10, 186, 181}
\usepackage{fancyhdr}
\pagestyle{fancy}
\fancyhf{}
\fancyhead[R]{\href{https://www.koito.co.jp}{\includegraphics[height=0.7cm]{../figures/IMG_3840(1).jpeg}}} % ロゴ画像のパスを変更してください
\fancyfoot[C]{\thepage}
\renewcommand{\headrulewidth}{0pt}

% リンクをティファニーブルー+下線にするコマンド
\newcommand{\uref}[1]{\hyperref[#1]{\textcolor{TiffanyBlue}{\underline{\ref*{#1}}}}}

% 定理環境
\theoremstyle{definition}
\newtheorem{theorem}{定理}[section]
\newtheorem{definition}[theorem]{定義}
\newtheorem{lemma}[theorem]{補題}
\newtheorem{proposition}[theorem]{命題}
\newtheorem{corollary}[theorem]{系}
\newtheorem{example}[theorem]{例}
\newtheorem{remark}[theorem]{注意}
\newtheorem{tabun}[]{多分}

\renewcommand{\proofname}{証明}
\makeatletter
\renewenvironment{proof}[1][\proofname]{%
  \par\pushQED{\qed}%
  \normalfont \topsep6\p@\@plus6\p@\relax
  \trivlist
  \item[\hskip\labelsep\bfseries#1\@addpunct{.}]\ignorespaces
}{%
  \popQED\endtrivlist\@endpefalse
}
\makeatother



\begin{document}
\begin{abstract}
    円分多項式に関するあれこれ
\end{abstract}

\section{円分体}
\begin{definition}
    $\zeta^m=1$を満たす元$\zeta$を
    $1$\textbf{の}$m$\textbf{乗根}という.
    $1$の$m$乗根という.
    $1$の$m$乗根$\zeta$で$1\leq d<m$なる
    整数$d$について$\zeta^d\neq 1$となるものを
    $1$\textbf{の原始}$m$\textbf{乗根}という.
\end{definition}

\begin{lemma}
    $\Omega$を代数的閉体とするとき次は同値である.
    \begin{enumerate}
        \item $1$の原始$m$乗根が存在する.
        \item $1$の$m$乗根全体が位数$m$の巡回群をなす.
        \item $\Omega$の標数を$p$とするとき$p$は$m$と互いに素である.
    \end{enumerate}
\end{lemma}

\begin{proof}
    $2\Rightarrow 1$:巡回群の生成元が$1$の原始$m$乗根である.
    $1\Rightarrow 3$:対偶を示す.$m=pn\, (n\in \mathbb{N})$と表すことができる.
    $\zeta$を$m$乗根とすると$\zeta^m=\zeta^{pn}=(\zeta^n)^p=1$である.
    $X^p-1=(X-1)^p$であることから$\zeta^n=1$となり,
    $1$の原始$m$乗根は存在しない.
    $3\Rightarrow 2$:
    $f(X)=X^m-1$の根を考えるとこれは$1$の$m$乗根である.微分$f'(X)=mX^{m-1}$は仮定より$0$でない.
    よって$f$は分離多項式であり,
    $f(X)$の根全体の集合を$G$とおくと
    $G$は位数$m$の群である.$d|m$となる整数$d$に対して,
    位数が$d$の$G$の部分群$H$の個数を考える.
    このような$H$が存在すると仮定して,$H$の任意の元$a$は
    フェルマーの小定理より$a^d=1$である.
    すなわち$g(X)=X^d-1$の根である.位数$d$の異なる部分群$H'$を
    考えてもやはり$g(X)$の根となることから$H$の取り方は
    高々一通りしかなく,すなわち$G$は位数$m$の巡回群である.
\end{proof}

\begin{theorem}
    $F$を任意の体,$\overline{F}$を$F$の代数的閉包とする.$\zeta_m\in 
    \overline{F}$を$1$の原始$m$乗根とすれば,$F(\zeta_m)/F$は
    アーベル拡大であり,$\mathrm{Gal}(F(\zeta_m)/F)\subset 
    (\mathbb{Z}/m\mathbb{Z})^\times$が成り立つ.
\end{theorem}

\end{document}